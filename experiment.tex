\section{Experiment and Analysis}
\label{sec:experiment}

In this section, we introduce the measurements experiments setup and 
evaluate the process of probabilistic forecasting of channel state.

% Subsec Experiment design
\subsection{Measurements}
\label{subsec:measurements}

% 24 hours measurement introduction
We perform measurements in neighborhoods, campus, downtown bussiness office and 
urban bussiness office for 24 hours on weekdays. 
The locations we chosen for measurements are shown in Fig.~\ref{fig:measurements}.

% Experiment equipment
We employ a Rohde \& Schwarz FSH8 portable spectrum works from 100 KHz to 8 GHz. 
The portable spectrum analyzer is controlled by a Python script on a laptop to measure 
the received signal strength.
% Data normalize 
To the best of our knowledge, there is no readily available mobile, multiband antenna from
450 MHz to 5.2 GHz on the market. Thus, we use a 700-MHz mobile antenna to perform in-field
measurements. We then normalize the mobile antenna performance across bands with indoor 
experimentation. To do so, we use a Universal Software Radio Peripheral (USRP) N210 to 
generate signals at 450 MHz, 800 MHz, and 2.4 GHz. We feed the USRP signals directly
to a spectrum analyzer and adjust the configuration of USRP to make the received signal 
strength the same as the 5.2 GHz signal from Gateworks 2358 with a XR5 radio. Then, we connect 
the signal source to a fixed multiband antenna (QT 400 Quad Ridge Horn Antenna) and measure the
received signal at a fixed distance with the 700 MHz antenna and antennas for different bands
to obtain the antenna loss for each band. We adjust the received signal strength
collected via the 700-MHz mobile antenna according to the normalization.

% Introduce how to calculate the capacity
% Explain multiband and activity level
When wireless devices operate in WiFi bands, the channel separation is relatively 
small (e.g., 5 MHz for the 2.4 GHz band). As a result, many works assume that
the propagation characteristics across channels are similar. However, with the
large frequency differences between WiFi and white space bands (e.g., multiple GHz),
propagation becomes a key factor in the deployment of wireless networks with both bands.
Here, a frequency band is defined as a group of channels which have
little frequency separation, meaning they have similar propagation characteristics.
In this work, we consider the diverse propagation and activity characteristics
for four total frequency bands: 450 MHz, 800 MHz, 2.4 GHz, and 5.2 GHz.
We refer to the two former frequency bands as white space bands and
the two latter frequency bands as WiFi bands.
The differences in propagation and spectrum utilization create opportunities
for the joint use of white space and WiFi bands in wireless access networks according
to the environmental characteristics (e.g., urban or rural and downtown or residential)
of the deployment location.


For spectrum utility and resulting channel availability, 
we split the measurements every 30 minutes of each band.  We define the percentage of sensing 
samples ($S_\theta$) above an interference threshold ($\theta$) over the total samples ($S$) in 
a time unit as the activity level ($A$) of inter-network interference:
\begin{equation}
\label{eq:actdef}
A=\frac{S_\theta}{S_a}
\end{equation}
The capacity of a clean channel is denoted by $C$. With the protocol model, the capacity 
of a channel with inter-network interference $C_r$ could be represented as 
the remaining free time of the channel capacity according to: 
\begin{equation}
\label{eq:intercap}
C_r=C*(1-\bar{A})
\end{equation}


% Need to discuss the \mu mapping with these measurements stuff



\subsection{Experiment Setup}
\label{subsec:experimentsetup}
















\subsection{Results and Analysis}
\label{subsec:results}


