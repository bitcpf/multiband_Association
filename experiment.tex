\section{Experiment and Analysis}
\label{sec:experiment}

In this section, we introduce the measurements experiments setup and 
evaluate the process of probabilistic forecasting of channel state.

% Subsec Experiment design
\subsection{Experiment Setup}
\label{subsec:experimentdesign}

% 24 hours measurement introduction
We perform measurements in neighborhoods, campus, downtown bussiness office and 
urban bussiness office for 24 hours on weekdays. 


% Experiment equipment
We employ a Linux-based 802.11 testbed, which includes a Gateworks 2358 board with 
Ubiquiti XR radios (XR9 at 900 MHz, XR2 at 2.4 GHz, XR5 at 5.2 GHz) and a DoodleLabs DL475 
radio at 450 MHz.  We develop shell scripts which utilize tcpdump to enable the testbed to
work as a sniffer, recording all 802.11 packets. However, since the Gateworks platform only 
updates its estimate of received signal strength upon the reception of a new packet (and
not all relevant channel activity is 802.11 based), we employ a spectrum analyzer to form 
a notion of inter-network interference with finer granularity.  Hence, we also use a Rohde \& Schwarz FSH8 
portable spectrum works from 100 KHz to 8 GHz. The portable spectrum analyzer is controlled 
by a Python script on a laptop to measure the received signal strength.

%We perform experiments in downtown Dallas, SMU campus, and neighborhood. The results show
%no 802.11 packets detected in white space bands(450 MHz, 900 MHz) there. 
%And in DFW area, as far as we know, we are the only group holds FCC license of white space bands. 
%Our experiments verify that these bands have not been used for commercial wireless data communication.
%Moreover, we observed that Gateworks platform only update its received signal strength when received
%a new packet.
%It is not good for inter-network interference measurement. To cover the gap,
%we employ a spectrum analyzer, multiband antenna, mobile antenna and a laptop developing
%a spectrum sensing system.

% Data normalize 
To the best of our knowledge, there is no readily available mobile, multiband antenna from
450 MHz to 5.2 GHz on the market. Thus, we use a 700-MHz mobile antenna to perform in-field
measurements. We then normalize the mobile antenna performance across bands with indoor 
experimentation. To do so, we use a Universal Software Radio Peripheral (USRP) N210 to 
generate signals at 450 MHz, 800 MHz, and 2.4 GHz. We feed the USRP signals directly
to a spectrum analyzer and adjust the configuration of USRP to make the received signal 
strength the same as the 5.2 GHz signal from Gateworks 2358 with a XR5 radio. Then, we connect 
the signal source to a fixed multiband antenna (QT 400 Quad Ridge Horn Antenna) and measure the
received signal at a fixed distance with the 700 MHz antenna and antennas for different bands
to obtain the antenna loss for each band. We adjust the received signal strength
collected via the 700-MHz mobile antenna according to the normalization.



% Introduce how to calculate the capacity

