\begin{abstract}
% White space
The FCC has reapportioned spectrum from TV white spaces for the purposes of large-scale Internet 
connectivity via wireless topologies of all kinds. 
% Limitations
These frequency resource offers additional channel capacity and propagation flexibility for network design.
The far greater range of these lower carrier frequencies are especially critical in user mobility 
adaptation, where high levels of aggregation could dramatically lower the power consumption of 
network operating. 
However, the white spaces resource distribution is restricted by FCC contrast to the population density, 
dense area has few white space channels.
Thus, leveraging the range of spectrum across user mobility becomes a critical issue for the operating of 
heterogeneous data networks with WiFi and limited white space bands. 
% Work
In this paper, we present a feasibility analysis for a heterogeneous network resource allocation to reduce 
the power consumption via queuing theory approach.
In particular, we study the spectrum utility across multi-bands and the user mobility across a weekday in 
typical environment of the Dallas-Fort Worth metroplex through measurements. 
Moreover, we propose a Greedy Server-side Replace (GSR) algorithm to reduce the power consumption with 
white space channels application. 
In doing so, we find that networks with white space bands reduce the power consumption by up to 512.55\% in sparse 
rural area over WiFi-only solutions via measurements driven numerical simulation. In more populated areas. we find 
an power consumption reduction on average across 24 hours by 24.57\%, 46.27\%,67.40\% over WiFi only network with 
one to three white space channels respectively.
We further investigate the quality of service requirements impacts on power consumption across the number of channels.
We find that the power consumption reduction is up to 150.89\% with three white space channels in dense areas with 
relaxed required waiting time.
\end{abstract}
