\begin{abstract}

%While many metropolitan areas sought to deploy city-wide WiFi networks, the densest urban areas were not
%able to broadly leverage the technology for large-scale Internet access.  Ultimately, the small 
%spatial separation required for effective 802.11 links in these areas resulted in prohibitively large up-front 
%costs.  

The FCC has reapportioned spectrum from TV white spaces for the purposes of large-scale Internet 
connectivity via wireless topologies of all kinds. 
The far greater range of these lower carrier frequencies are especially critical in dense areas, where 
high levels of aggregation could adapt the mobility of users.
However, the restriction of white space in dense area limits the available number of white space channels 
in dense areas. 
Thus, leveraging the range of spectrum across user mobility becoms a critical issue for the deployment of 
data networks with WiFi and white space bands. 
% Work
In this paper, we measure the spectrum utility in typical environment of the Dallas-Fort Worth metropolitan. 
We formulate the heterogeneous wireless network with both WiFi and white space bands as a queuing system. 
Further, we propose a measurement-driven resource allocation algorithm, Greedy Server-side Replace (GSR).
In particular, we study the white space and WiFi bands with in-field spectrum utility measurements, revealing 
the power consumption required for an area with channels in multiple bands. In doing so, we find that 
networks with white space bands in dense areas reduce the power consumption by up to FIXME 
% In sparse area, the power consumption is the most


\end{abstract}
