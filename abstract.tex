\begin{abstract}
% White space
Cellular networks have addressed the multi-fold increase in traffic demand through
various approaches from increasingly smaller cells to offloading demand to unlicensed
spectrum, such as WiFi and TV white spaces. The latter approach has a tremendous 
cost benefit as unlicensed hardware can be co-located with existing cellular infrastructure.
However, in these situations where demand is the greatest, WiFi spectral 
activity could be high and the total number of available white space
channels are often inversely proportional to the population due to the existence of TV 
broadcast channels. Moreover, the additional hardware induces a higher energy demand 
of the cell site. In this work, we perform extensive spectral measurements of unlicensed
bands in a major metropolitan area and crowd-source user mobility to consider
potential swings in network load. Using these measurements, we design a queuing-based 
approach to serve these network demand dynamics in an energy-efficient manner according
to differing qualities of service. In particular, we perform in-field experimentation 
of the diurnal spectrum utility across white space (54-806 MHz) and WiFi (2.4 and 5.8 GHz) 
bands in typical settings across the Dallas-Fort Worth metroplex. We also consider mobility 
patterns from Android-based crowd-sourced measurements in four major Texas cities to infer the 
change in traffic demand induced by users. Driven by both data sets, we propose a 
Greedy Server-side Replace (GSR) algorithm to estimate the power consumption for the use 
of unlicensed bands for cellular offloading. In doing so, we find that networks with white 
space bands reduce the power consumption by up to 512.55\% in sparse rural areas over 
WiFi-only solutions via measurements driven numerical simulation. In more dense areas, 
we find power consumption reduction across a 24-hour period to be, on average, 
24.57\%, 46.27\%, 67.40\% over WiFi-only offloading with one to three white space channels, 
respectively. We further investigate the quality of service requirements impact on power 
consumption. Finally, we find that the power consumption reduction is up to 150.89\% 
over a WiFi-only configuration with three white space channels in dense urban areas with 
a relaxation of waiting times.
\end{abstract}

%The FCC has reapportioned spectrum from TV white spaces for the purposes of 
%large-scale Internet connectivity via wireless topologies of all kinds. 
%These low frequency resource offers propagation flexibility for network design 
%as well as additional channel capacity.
%The far greater range of these lower carrier frequencies are especially critical 
%in user mobility adaptation, where high levels of aggregation could dramatically 
%lower the power consumption of network operating. 
%% Limitations
%However, the white spaces resource application is restricted by FCC contrast to 
%the population density, dense area has few white space channels.
%Thus, leveraging the range of spectrum across user mobility becomes a critical 
%issue for the operating of heterogeneous data networks with WiFi and limited white 
%space bands. 
%% Work
%In this paper, we present a feasibility analysis for a heterogeneous network 
%resource allocation to reduce the power consumption via a queuing theory approach.

