% Background and problem model
\subsection{Challenges And Analysis}
\label{subsec:challenge}

Prior works model similar multi-channel system as $M/M/m$ queuing system for analyzing.
Other than prior works, this system is not able to be formulated as a $M/M/m$ queuing system due to the 
propagation variation across white space bands and WiFi bands. The users in this system can only be 
served by the white space channels and the WiFi channels assigned in the cell. 
Moreover, we use a measurement based channel estimation other than an on-off channel model in 
this work. 
To design such wireless system, the waiting time constraints have to be satisfied for the quality of service. 
Previous multi-channel works apply $M/M/m$ queuing model for similar system. However, the spectrum 
variation and the white space channel splitting in this system remove the equal service capacity in the system.
Thus, $M/M/m$ queuing system is not directly applicable to get the minimum channel resource which is the
number of servers of the queuing system. 

% User diversity
Also a practical system suffer multi-user diversity.
Multi-user diversity is a form of diversity inherent in a wireless network, provided by 
independent time-varying channels across the different users~\cite{viswanath2002opportunistic}.
% Channel capacity diversity
The user diversity make the channel capacity vary across users. In this system, some WiFi cells may 
have good white space channels while the others may suffer worth white space performance. 
Thus, in-field measurements based channel capacity estimation become an important role in such system.


To answer the design question, we first get the channel capacity from the in-field measurement, then we 
propose a Greedy Server-side Replace (GSR) algorithm to minimize the power consumption and achieve the 
performance constraints.


% User group
The users of the the same WiFi cell are in a heterogeneous queuing system with server of 
WiFi channels and white space channels with service rate $\mu_1,\mu_2,...\mu_{(F_w+1)}$.
The white space channel capacity could be split into multiple WiFi cells, thus, the 
capacity of the white space is usually the minimum channel capacity in a WiFi cell.
The white space channels in a WiFi cell should be less than the WiFi channel or about equal. 
Thus, we apply the transformation model in~\cite{yu2008transformation} to estimate the response 
time $\bar{w}$ which should be less than $W$.
%\label{eq:transformation}
%\mu_{min}=\min{(\mu_1,\mu_2,...\mu_{(F_w+1)})} \\
%\mu_{max}=\max{(\mu_1,\mu_2,...\mu_{(F_w+1)})}//
%k=\frac{\mu_{max}}{\mu_{min}}
%\end{equation}
In this transformation model, the actual arrival rate for one specific server $\lambda_s$ is defined as in 
Eq.~\ref{eq:actarrivalrate}
\begin{equation}
\label{eq:actarrivalrate}
\lambda_s=D/(F_w+1)
\end{equation}

The other parameters are defined in Eq.~\ref{eq:transformation} to~\ref{eq:kvalue}.
\begin{equation}
\label{eq:transformation}
\mu_{min}=\min{(\mu_1,\mu_2,...\mu_{(F_w+1)})} = \bar{\mu}
\end{equation}

\begin{equation}
\mu_{max}=\max{(\mu_1,\mu_2,...\mu_{(F_w+1)})} 
\end{equation}

\begin{equation}
\label{eq:kvalue}
k= \lfloor\frac{\mu_{max}}{\mu_{min}} \rfloor
\end{equation}

When $k=1$, the system could be treated as a homogeneous system. Otherwise,   
$k\ge2$ the average response time of such heterogeneous system could be represented as in Eq.~\ref{eq:heterresponse}\cite{yu2008transformation}:

\begin{equation}
\label{eq:heterresponse}
\bar{w}=\frac{1}{\frac{1}{3}\bar{\mu}(2k+1)-\lambda_s}
\end{equation}

If the traffic could be served only by part of a single white space channel or the WiFi channel, the users in the cell 
converge into a $M/M/1$ queuing. Another scenario is when the traffic occupies two or more channels, the system become 
a $M/M/m$ queuing system. The average response time is calculated as in Eq.~\ref{eq:mm1w} and Eq.~\ref{eq:mmmw}~\cite{gelenbe1998introduction}.
\begin{equation}
\label{eq:mm1w}
\bar{w}=\frac{1}{\mu^+-D}
\end{equation}
$\mu^+$ is the channel capacity of the single server in queuing system.
\begin{equation}
\label{eq:mmmw}
\bar{w} = \frac{1}{\mu^*}(1+\frac{c(m,\rho)}{m(1-\rho)})\approx \frac{1}{\mu^*}\frac{1}{1-\rho^m}
\end{equation}
$\mu^*$ is the channel capacity in the $M/M/m$ queuing system.
$\rho=\frac{\lambda}{m\mu^*}$ is the traffic density, and $c(m,\rho)$ is the Erlang-C formula~\cite{gelenbe1998introduction}.

Through the above analysis, we could calculate the channel capacity need to be spent for certain group of users. 
% Here
% Do I need to consider the rate of the consumption?
To model the power consumption of the system, we assume the power consumption of each operating radio include 
standby and transmitting power.
The operated radios $R_i$ in the system is chosen from the WiFi radios $F_M$ and the white space radios $F_w$. 
If the a radio $R_i$ is chosen, $R_i = 1$, otherwise $R_i = 0$, as shown in Eq.~\ref{eq:radio}
%FIXME need to clarify the \mu, may replace the H with mu

\begin{equation}
\label{eq:radio}
 R_i(t) = \left\{ 
	  \begin{array}{l l}
	    P_s+P_t\cdot\mu   &  \sum\limits_{j=1}^N A_{i,j}(t) \ge 1\\
		0 &  Otherwise
			    \end{array} \right.
\end{equation}


For the carrier, reduce the power consumption is important for cost saving.
Our goal is, during a certain $T$ time slot, to minimize the power consumption as represented in Eq.~\ref{eq:objective}:
\begin{equation}
\label{eq:objective}
R^*(t) = \min{\{\sum\limits_{t=0}^{T}\sum\limits_{i=1}^{(F_M+F_w)} R_{i}}(t)\}
%\xi = \min{\{\sum\limits_{t}^{t+T}\sum\limits_{i}^{(F_M+F_w)} R_{i}}(t)\}
\end{equation}
$R^*$ represent the minimum operating radios power during $T$ time slots required to satisfy the 
users in the system. 

In this model, the less radio we operate in each time slot, the less power consumption will be cost in the system.
To split the white space capacity we propose a Greedy Server-side Replace(GSR) algorithm to approach the minimize 
number of radios operate in the system. The algorithm output the channel resource allocation to satisfy the waiting 
time constraint of the system.


% power consumption extreme
For the power consumption, there are two extremes. When the traffic rate $D$ is low, a white space channel could 
serve all the users in the propagation range. On the other side, when all the WiFi cells have high traffic demand, 
both the WiFi and white space radios have to work for the users. Between the two extremes, when the user traffic 
is medium heavy and the location distribution is non-uniform, the white space channels could adapt these cases.

A white space radio is able to replace a WiFi radio if part of the channel capacity could serve the users in the 
WiFi cell. If each WiFi cell could be served by the least channel resource, the more radios could be turned off. 
The less capacity assigned for the WiFi cells, the less power will be cost for transmission.

\begin{algorithm}[t]
\small
\caption{Greedy Server-side Replace}
\label{algorithm:mape}
\begin{algorithmic}[1]
\REQUIRE  ~~\\
$N$: Users\\
$H_{i,j}^f$: Vector of channel capacity\\
$D$: Traffic Rate\\
$n$: Path Loss Exponent \\
$M$: WiFi Cells
\STATE {Find the WiFi cell with the lowest traffic rate $D$,break the tie with index}
\IF {Check with Eq.~\ref{eq:heterresponse}~\ref{eq:mm1w}~\ref{eq:mmmw} if white space channel satisfy the users, start with the worst/most utilized white space channel}
\STATE Replace the WiFi channel with the white space channel
\STATE Apply half-interval search to find the minimum capacity for the users
\ELSIF {The WiFi channel satisfy the users}
\STATE Keep the WiFi channel and find the minimum capacity for the users
\ELSIF {White space channels available}
\STATE Adding white space resource to the cell
\ELSE 
\STATE Get the response time of the cell with all available resource
\ENDIF
\STATE Update the system information
\STATE Repeat the process for all the cells
\STATE Calculate the power consumption
\ENSURE ~~\\
The power consumption and the maximum response time\\
\end{algorithmic}
\end{algorithm}

The calculation of channel capacity $H$ will be discussed in Section~\ref{sec:experiment}.
The algorithm starts to reduce the standby power then the transmitting power for each server.
%
%
%% Traffic served
%With a schedule $A$, the served traffic flow of a user is the minimize value of the queue with 
%enough channel capacity, or the remaining capacity. The condition $\psi$ as shown in Eq.~\ref{eq:sfcondition}
%\begin{equation}
%\label{eq:sfcondition}
%		\psi =  \sum\limits_{j=1}^N \gamma_j(t) \cdot A_{i,j}(t) 
%		%when \ \sum\limits_{j=1}^N \gamma_j(t) \cdot A_{i,j}(t) > C
%		%	    \end{array} \right.
%\end{equation}
%
%The served traffic flow is represented as in Eq.~\ref{eq:effectiverate}
%\begin{equation}
%\label{eq:effectiverate}
% \gamma_j(t) = \left\{ 
%	  \begin{array}{l l}
%	     \min\limits_{j\in N}(Q_j(t),\sum\limits_{i=1}^{F_M+F_w} A_{i,j}(t) \cdot H_{i,j}(t)) & \psi \le C    \\ 
%		 0 & \psi > C
%			    \end{array} \right.
%\end{equation}
%
%
%
%% The length of the queue
%The new arrival traffic demand $D_i(t)$ can not be served at the current time slot $t$.
%The individual queues $Q_j(t),j\in N$ at the end of the time slot $t$ is represented as Eq.~\ref{eq:userqueue}:
%\begin{equation}
%\label{eq:userqueue}
%%Q_j(t) = Q_j(t-1)+D_j(t)-\min{\{\sum\limits_{i=1}^{(F_m+F_w)} H_{i,j}(t)\cdot A_{i,j}(t),Q_j(t-1)}\}
%Q_j(t) = Q_j(t-1)+D_j(t)- \gamma_j(t)
%\end{equation}
%
%
%
%% Here D is preknow in T slots and discuss the T constraints
%In this system, the channel capacity $H_{i,j}^f(t)$ is from the SNR according to 
%Friis model and in-field measurements. We consider discrete time with a suitably 
%chosen small time unit. %One user can only associate 
%%with a single channel during a time slot to avoid heavy interference. 
%Users scheduled with the same channel will share the time unit through time division inside 
%the time slot. 
%
%% Constraints
%
%
%% Discuss the feasibility of buffer
%The minimize number of operating radios has to fulfill the maximize capacity of each user greater than 
%the remaining buffer and the new coming data as shown in Eq.~\ref{eq:capacity}
%\begin{equation}
%\label{eq:capacity}
%\sum\limits_{j=1}^{F_m+F_b}H_{ij}(T)\ge Q_i(T) ,\ i\in N
%\end{equation}
%Otherwise the buffer will be overflow, and there is no feasible solution of the system.




% Linear is not good enough complexity unsure channel state


% Greedy algorithm








% Channel state estimation








% Reduce the radio/server/channel resource



% New section feasibility, and Max flow formulation

%The model could be formulated as a {\it Max-Flow} problem~\cite{networkoptimization}. The question is 
%looking for a feasible flow $\Gamma$ under the constraints of channel capacity $H$ and traffic demand $D$. 
%The lower bound of flow is $0$, the upper bound of a single flow from the accss point $i$ to user $j$ is 
%the measurements based $H_{i,j}^b$.
%Thus, $\gamma_j \le \sum\limits_{i\in M} \sum\limits_{b \in B} H_{i,j}^b\cdot A_{i,j}^b$. The max-flow model 
%of the problem is feasible. 











% Greedy algorithm








% Channel state estimation








% Reduce the radio/server/channel resource



% New section feasibility, and Max flow formulation

%The model could be formulated as a {\it Max-Flow} problem~\cite{networkoptimization}. The question is 
%looking for a feasible flow $\Gamma$ under the constraints of channel capacity $H$ and traffic demand $D$. 
%The lower bound of flow is $0$, the upper bound of a single flow from the accss point $i$ to user $j$ is 
%the measurements based $H_{i,j}^b$.
%Thus, $\gamma_j \le \sum\limits_{i\in M} \sum\limits_{b \in B} H_{i,j}^b\cdot A_{i,j}^b$. The max-flow model 
%of the problem is feasible. 

