% Background and problem model
\subsection{Challenges And Analysis}
\label{subsec:challenge}
% Challenges sum
Prior works formulate similar multi-channel system as $M/M/m$ queuing system for analyzing~\cite{bodas2012low}.
However, this system is not able to be formulated as a $M/M/m$ queuing system due to the 
non-equal capacity of the assigned channel capacity across white space and WiFi channels.
Thus we first analyze the queuing system and apply previous work in $M/M/m$ queuing theory to 
generate the solution for such system.

% introduce the queue system
Without the white space channels, the users in a cell can only access to the WiFi channel assigned for the 
cell to get service. The large propagation of white space channels offers more options for all the users. 
The user could either associate with the white space channels or the WiFi channels. 
Thus, the white space channels could be slitted into several cells. 
The splitting of the white space channels brings more capacity variation of the servers. 
The white space channel splitting capacity variation and the spectrum capacity variation of cells might remove the 
equal service capacity assumption of the system in most scenarios.
However, the equal server capacity is the pre-request for a general $M/M/m$ queuing system analysis.
Thus, the $M/M/m$ queuing system of a multi-channel version is not directly applicable for this system model.



% User constraints
In the design of the wireless system, the response time $W$ constraints have to be satisfied to keep the 
quality of service. 
The users in this system can only be served by the large propagate white space channels or the WiFi 
channels assigned in the cell. 
% User diversity
The users located in multiple cells have different channel status in the same white space channels, 
which is mentioned as part of the multi-user diversity in previous works. Multi-user diversity is a 
form of diversity inherent in a wireless network, provided by independent time-varying channels across 
the different users~\cite{viswanath2002opportunistic}. 
The diversity could be generated by the interference from device inside the network or out of the network, 
the environmental variations.  
% Channel capacity diversity
The variation among users make the channel capacity among all the cells for white space channels. 
Thus, some cells may have clean white space channels in the air while the other cells may suffer 
worse white space channel performance. 
To address the variation, instead of holding the on-off channel assumption, we implement an in-field 
measurements based channel capacity estimation approach in this work.

% Activity level instruction
We define an activity level to estimate the achieved channel capacity. 
We perform measurements to sense the activities in the air through our portable spectrum analyzer.
The percentage of sensing samples ($S_\theta$) above an interference threshold ($\theta$) over the 
total samples ($S$) in a time unit is the activity level ($A$):
\begin{equation}
\label{eq:actdef}
A=\frac{S_\theta}{S_a}
\end{equation}
The capacity of a clean channel is denoted by $C$. With the protocol model, the achieved capacity 
of a channel $C_r$ could be represented as the remaining free time of the channel capacity 
according to Eq.~\ref{eq:intercap}: 
\begin{equation}
\label{eq:intercap}
C_r=C*(1-\bar{A})
\end{equation}

% User distribution measurements
Other than the achieved channel capacity, we also perform in-field measurements of the user mobility footprint. 
When the total number of the users is a certain value, the user distribution becomes important for wireless network 
operating. We analyze the data set from WiEye, an Android application reports the location, velocity and signal 
information to leverage the mobility pattern of users during week days. The setup and results are shown in 
Section~\ref{sec:experiment}.

% analysis
With these measurements information, we further analyze the channel capacity allocation for such a system. 
We first investigate the channel capacity allocation in a single cell.
The users of the the same WiFi cell are in a heterogeneous queuing system with server of 
WiFi channels and white space channels with service rate $\mu_1,\mu_2,...\mu_{(F_w+1)}$.
$\mu$ denote the capacity allocated for this cell.
The white space channel capacity is slitted into multiple WiFi cells, thus, the 
capacity of the white space is usually the minimum channel capacity in a cell.
Thus, there are three cases of the channel capacity in a single cell. 
The first scenario is several channels of both WiFi and white space works for the cell and the capacity of some 
white space channel is several time less than other channel capacity since white space channels are slitted 
for many cells. 
An example here is in a single cell, the white space channel assigned is equally distributed to two cells, while 
the WiFi channel works in this cell, thus, the capacity of the WiFi channel is about twice of the white space 
channel capacity in this cell.
The second scenario is only the WiFi channel or only part of a single white space channel works 
for the cell. 
The third scenario is several channels work for this cell with around equal capacity.
The fist case is a heterogeneous server queuing system with unequal capacity servers.
The second case is simplified as a $M/M/1$ queuing system. 
The third case is converted into a $M/M/m$ queuing system.

For the first case heterogeneous server queuing system, we apply the transformation model 
in~\cite{yu2008transformation} to estimate the response time $\bar{w}$. 
In the transformation model, the actual arrival rate for one specific server $\lambda_s$ is 
defined as in Eq.~\ref{eq:actarrivalrate}
\begin{equation}
\label{eq:actarrivalrate}
\lambda_s=D_{cell}/(F_w+1)
\end{equation}
$D_{cell}$ is the traffic aggregated from the users in the cell.

The other parameters are noted from Eq.~\ref{eq:transformation} to~\ref{eq:kvalue}.
\begin{equation}
\label{eq:transformation}
\mu_{min}=\min{(\mu_1,\mu_2,...\mu_{(F_w+1)})} = \bar{\mu}
\end{equation}

\begin{equation}
\mu_{max}=\max{(\mu_1,\mu_2,...\mu_{(F_w+1)})} 
\end{equation}

\begin{equation}
\label{eq:kvalue}
k= \lfloor\frac{\mu_{max}}{\mu_{min}} \rfloor
\end{equation}

When $k=1$, the system becomes a homogeneous queuing system as in case three. Otherwise,   
$k\ge2$ the average response time of such heterogeneous system~\cite{yu2008transformation} 
could be represented as in Eq.~\ref{eq:heterresponse}:

\begin{equation}
\label{eq:heterresponse}
\bar{w}=\frac{1}{\frac{1}{3}\bar{\mu}(2k+1)-\lambda_s}
\end{equation}

Through the transformation model, we could further calculate the channel capacity required for 
response time constraints. Further, the power consumption could be calculated for the system.
When the traffic could be served by part of a single white space channel or the WiFi channel, as 
the second scenario, the system converge into a $M/M/1$ queue. The response time $\bar{w}$ 
could be estimated from Eq.~\ref{eq:mm1w}~\cite{gelenbe1998introduction}.

\begin{equation}
\label{eq:mm1w}
\bar{w}=\frac{1}{\mu^+-D}
\end{equation}
$\mu^+$ is the channel capacity of the single channel capacity in queuing system.

% Here
In the third scenario, the system could be treated as a $M/M/m$ queuing system. 
The average response time is calculated as in Eq.~\ref{eq:mmmw}~\cite{gelenbe1998introduction}.
\begin{equation}
\label{eq:mmmw}
\bar{w} = \frac{1}{\mu^*}(1+\frac{c(m,\rho)}{m(1-\rho)})\approx \frac{1}{\mu^*}\frac{1}{1-\rho^m}
\end{equation}
$\mu^*$ is the average capacity of channels in the $M/M/m$ queuing system.
$\rho=\frac{\lambda}{m\mu^*}$ is the traffic density, and $c(m,\rho)$ is the Erlang-C 
formula~\cite{gelenbe1998introduction}.



In this model, the less radio in operation and the less power consumption will be cost in the system according to 
Eq.~\ref{eq:radio}. 
The basic idea of power reduction for this heterogeneous system is to replace the WiFi radios via white space channel 
capacity. To implement the division of the white space capacity we propose a Greedy Server-side Replace(GSR) algorithm 
to approach the minimize power consumption in the system as shown in Algorithm~\ref{algorithm:gsr}. 

\begin{algorithm}[t]
\small
\caption{Greedy Server-side Replace}
\label{algorithm:gsr}
\begin{algorithmic}[1]
\REQUIRE  ~~\\
$N$: Users\\
$H_{i,j}^f$: Vector of channel capacity\\
$D$: Traffic Rate\\
$M$: WiFi Cells
\STATE {Find the WiFi cell with the lowest traffic rate $D$,break the tie with index}
\STATE {Calculate the power consumption according to Eq.~\ref{eq:heterresponse}~\ref{eq:mm1w}~\ref{eq:mmmw}}
\IF {If channel resource feasible and exist unserved traffic demand}
\STATE {List available options}
\IF {Single channel is the best}
\STATE Apply half-interval search to find the minimum capacity for the users
\ELSIF {Homogeneous is the best}
\STATE Allocate the resource for the cell
\STATE Keep the WiFi channel and find the minimum capacity for the users
\ELSIF {Heterogeneous is best}
\STATE Adding white space resource to the cell
\ENDIF
\ELSE 
\STATE Get the waiting time of the cell with all available resource
\ENDIF
\STATE Update the system information
\STATE Repeat the process for all the cells
\STATE Calculate the power consumption
\ENSURE ~~\\
The power consumption, resource allocation and the maximum waiting time\\
\end{algorithmic}
\end{algorithm}

The algorithm input the measurement-based achieved channel capacity, the number of white space channels and the WiFi cells. 
The white space channels are employed for the cell with less traffic demand to reduce the standby power. The algorithm 
compare the three configurations of channel capacity and tell the setup with the best power consumption. 
Further, the process is repeat till all the traffic demand are served or the channel resource has been used up.
Then output the power consumption and channel allocation of the system.



