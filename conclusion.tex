\section{Conclusion}
\label{sec:conclusion}
In this paper, we considered the use of a small number of white space channel resources for reducing power consumption across a broad range of user mobility patterns and population densities. 
Towards this goal, we propose a heterogeneous wireless network and model it as a queuing system for additional analysis based on previous queuing theory work. 
To efficiently utilize the white space channels, we proposed a Greedy Server-side Replacement algorithm to allocate resources such that the power consumption in the network is minimized. 
We then performed spectrum utilization measurements in the several distinct locations including a downtown area, urban neighborhood, campus, and suburban neighborhoods to evaluate our algorithm. 
% Fixme
We leverage the impacts of channel quality, user mobility, quality of service, and population density on the network offloading structure. 
Further we investigate the performance in a virtual city to simulate the expected in-field performance. 
Through this simulation, we show that using white space channels can greatly reduce the power consumption of such offloading systems in most scenarios. 
From our results, we also see the advantage of using white space channels in adapting user mobility to reduce power consumption. 
% end
Through extensive analysis of spectrum utilization and user mobility, we showed that white space bands can reduce the average power consumption in white space heterogeneous networks by 64.70\% on average over 24 hours. 
Finally, we integrated previous measurements work in north Texas and find the power consumption is reduced by white space bands by 512.55\% in sparse area.
In the future, we will extend our work to include large scale user mobility patterns in addition to spectral activity levels for heterogeneous wireless network deployment optimization.
% Future work

