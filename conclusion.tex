\section{Conclusion}
\label{sec:conclusion}
In this paper, we considered the use of few white space channel resource 
for improving the power consumption across a broad range of user mobility, 
population densities. To consider the power consumption of a system, we 
formulate the heterogeneous wireless network as a queuing system. 
We then analyze the heterogeneous queuing system based on previous queuing 
theory work. To address the resource allocation of white space channels, 
we proposed a Greedy Server-side Replace algorithm to find the resource allocation 
with minimum power consumption. We then perform spectrum utilization 
measurements in the typical areas of downtown, urban, campus, neighborhoods 
to drive the algorithm. 
% Fixme
We also leverage the impacts of channel quality, user mobility, quality of service,
and population density on power consumption.
% end
Through extensive analysis across the spectrum utilization 
and user mobility, we show that white space bands can reduce the average of 
power consumption by 64.70\% on average over 24 hours. We also integrated previous 
measurements work in north Texas and find the power consumption is reduced by white space
bands by 512.55\% in sparse area.
In the future, we will consider to propose the heterogeneous wireless network 
deployment with large scale user mobility patterns.
% Future work

