\section{Introduction}
\label{sec:introduction}

%  Multiband background
The FCC has approved the use of broadband services in the white spaces of UHF TV bands, 
which were formerly exclusively licensed to television broadcasters. These white space 
bands are now available for unlicensed public use, enabling the deployment of wireless 
data networks. These white space bands operate in available channels from 54-806 MHz, 
having a far greater propagation range than WiFi bands for similar transmission power~\cite{balanis2012antenna}. 
% Improve existing cells
Thus, white space bands greatly complements the existing WiFi wireless network with 
their large propagation range for user mobility. 

% Multi-user diersity
The users in multiple locations under the coverage of both the WiFi and white space have user 
diversity which represents the difference of transmission qualities due to 
the variations in channel quality, geographic location, interference, etc.
% Explain the spectral diversity temporal diversity
The user diversity mainly comes from two types of reason. One is the temporal diversity which 
is caused by the environment variation. The other is the spectral diversity which represents 
the transmission conditions varies across frequencies. 
In some moderate number of users, the sum capacity of users increase with the diversity in 
the system~\cite{gan2014multiple}. 

The larger propagation range of white space channels is able to adapt channel association of 
users located in large area. 
The white space channels are able to serve the users distributed in a large area.
With both WiFi channels and white space channels, the users of an existing WiFi cell 
have the options to associate with either the WiFi channel assigned to the cell or the 
large propagation white space channels covering multiple cells.
When the users distributed in a large area, the temporal diversity and spectral diversity become 
key issues in white space wireless network applications. 
% Extreme cases and the middle of number of channels
In sparse rural areas, plenty of white space channels are able to deploy new white space network. 
However, in dense area, the white space channels are restricted by FCC in a small number.
For example, there is none white space channel available in New York downtown~\cite{googlespectrum}. 
Other than the plenty white space channels and none white space channel extreme cases, most areas 
of major cities in the United States have one to eight 
white space channels~\cite{googlespectrum}. 
Since the carrier have deployed WiFi wireless networks in most of the dense areas,
the white space channels are able to complement the WiFi wireless networks to achieve better 
performance in coverage, power consumption, etc. Thus, exploiting these limited white space resource 
to improve the WiFi network in dense area is a perspective option for wireless networks. 

% Power saving, extreme cases
The white space frequencies offer not only more wireless capacity but also the convenience of access 
across large area for the heterogeneous wireless network structure. 
A single white space channel is able to satisfy all the users in the area when the total traffic demands of 
the users in the area are relatively low. The WiFi radios in the heterogeneous structure could be turned off 
for power saving. 
While when the total traffic demand of the users in the area is relatively high, all the radios in the wireless network 
have to operate for the service of users.
However, the amount of traffic demands generally come across somewhere between these two extremes. Thus, the 
question comes out, {\it in what degree the white space help to reduce the power consumptions of an existing 
WiFi mesh?} Especially when the WiFi mesh located in dense area with less white space channel availability.

% Work summary 
In this work, we study the white space resource impacts on mesh network power consumption via queuing theory model.
We describe the heterogeneous wireless network structure with both white space bands and WiFi bands. 
We perform in-field measurement to leverage the channel utility and mobility footprint of Dallas Fort-worth metroplex.
We analyze the heterogeneous structure under the the waiting 
time constraint for resource allocation in multiple scenarios via the queuing model. We further propose a greedy 
to find the resource allocation minimizing the power consumption for the heterogeneous network structure. 
We then evaluate the algorithm, showing the power consumption gains across sparse and dense areas and 
analyze the impact of the number of white space and WiFi channels in these representative scenarios.
% On demand consideration

% Paper contributions
In particular, the main contributions of our work are as follows:
\begin{itemize}
\item We perform 24 hours in-field measurements in neighborhoods, campus, downtown business building, 
and urban business buildings. Through these in-field measurements, we estimate the achieved channel capacity of 
these typical area in north Texas.
\item We leverage the user mobility footprint through in-filed Android based WiEye measurements of Dallas area on weekdays. 
Through the measurements analysis, we tell the user distribution in multiple types of areas across 24 hours in weekdays. 
\item 
We formulate the heterogeneous wireless structure as a queuing system. Based on previous queuing theory works, we 
analyze the resource allocation of the system quality of service in waiting time. Based on the analysis, we propose a 
Greedy Server-side Replace (GSR) algorithm to allocate the channel resource to minimize the power consumption.
\item We perform measurement-driven numerical simulations to analyze various scenarios of channel resource and 
users distribution. Our results shows that the white space bands reduce the power consumption in
sparse area by up to 512.55\% in weekdays.
\end{itemize}

The rest of the paper is organized as follows. We describe the system and formulate the problem in~\ref{sec:problemformulation}.
Then, we present the queuing theory analysis and the Greedy Server-side Replace algorithm also in~\ref{sec:problemformulation}.
The measurements and measurement-driven numerical simulation is discussed in~\ref{sec:experiment}. Finally, we 
conclude our article in~\ref{sec:conclusion}. 
