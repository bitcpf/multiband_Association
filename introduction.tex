\section{Introduction}
\label{sec:introduction}

%  Multiband background
The FCC has approved the use of broadband services in the white spaces of UHF TV bands, 
which were formerly exclusively licensed to television broadcasters. These white space 
bands are now available for unlicensed public use, enabling the deployment of wireless 
data networks. These white space bands operate in available channels from 54-806 MHz, 
having a far greater propagation range than WiFi bands for similar transmission power~\cite{balanis2012antenna}. 
% Improve existing cells
Thus, white space bands greatly complements the existing WiFi wireless network with 
their large propagation range for user mobility. 
The users of an WiFi access point have the options to associate with either the WiFi channel 
of the cell or the large propagation white space channels.

% Multi-user diersity
The users in multiple locations under the coverage of both the WiFi and white space have user 
diversity which represents the difference of transmission qualities of multiple users due to 
their variations in transceiver design, geographic location, etc.
% Explain the spectral diversity temporal diversity
The user diversity comes from two types of reason. One is the temporal diversity which 
is caused by the environment variation. Another is the spectral diversity which represents 
the transmission conditions varies across frequencies. 
In some moderate number of users, the sum capacity of users increase with the diversity in 
the system~\cite{gan2014multiple}. 

The larger propagation range of white space channels adapt channel association of users located 
in large area. When the users distributed in a large area, the temporal diversity and spectral 
diversity become the key issues in white space applications. 
% Extreme cases and the middle of number of channels
In sparse rural areas, plenty of white space channels are able to deploy new white space network. 
However, in dense area, few white space channels are available for new network deployment, such as 
none white space channel is available in New York downtown~\cite{googlespectrum}. The carrier have 
to use WiFi channels to deploy wireless networks in the dense area without any available white space channel. 
Other than these two extreme cases, most areas of major cities in the United States have one to eight 
white space channels~\cite{googlespectrum}. Exploiting these limited white space resource to improve 
the WiFi network in dense area is a perspective option for wireless networks. 

% Power saving, extreme cases
Moreover, the white space frequencies offer more wireless capacity and convenience of access across large 
area. When the total traffic demands of the users are relatively low, a single white space channels satisfy 
all the user. Then, the WiFi radios could be turned off for power saving. On the other side, when the users 
have highly total traffic demand, all the radios, WiFi and white space, have to be operated to serve the users. 
However, traffic demands generally come across somewhere between these extremes. Thus, the question comes out, 
a{\it in what degree the white space help to reduce the power consumptions of an existing WiFi mesh?}
Especially when the WiFi mesh located in dense area with less white space channel availability.

% Work summary 
In this work, we study the white space resource impacts on mesh network power consumption via queuing theory model.
We describe the heterogeneous network structure with white space bands and WiFi bands and analyze the 
system with a queuing model. We analyze the heterogeneous structure to get resource required by the waiting 
time restriction in multiple scenarios of the structure. We further propose a Greedy Server-side Replace (GSR) 
algorithm to minimize the power consumption for the heterogeneous network structure. 
We then evaluate the algorithm, showing the power consumption gains across sparse and dense areas and 
analyze the impact of the number of white space and WiFi channel in these representative scenarios.
% On demand consideration

% Paper contributions
In particular, the main contributions of our work are as follows:
\begin{itemize}
\item We perform 24 hours in-field measurements in neighborhoods, campus, downtown business building, 
and urban business buildings. Through the measurements, we estimate the achieved channel capacity of 
these typical area in Dallas area.
\item We leverage the user mobility footprint through in-filed WiEye measurements of Dallas on weekdays. 
Through the measurements, we tell the user distribution in multiple types of areas. 
\item 
We formulate the heterogeneous wireless system as a queuing system. Based on previous queuing theory works, we 
analyze the resource relation of the system waiting time. Based on the analyze, we propose a Greedy Server-side Replace 
(GSR) algorithm to allocate the channel resource to minimize the power consumption.
\item We perform measurement-driven numerical simulation to analyze various scenarios of channel resource and 
users distribution. Our results shows that the white space bands reduce the power consumption in
sparse area by up to 512.55\% and 99.57\% in weekdays.
\end{itemize}

The rest of the paper is organized as follows. We describe the system and formulate the problem in~\ref{sec:problemformulation}.
Then, we present the queuing theory analysis and the Greedy Server-side Replace algorithm also in~\ref{sec:problemformulation}.
The measurements and measurement-driven numerical simulation is discussed in~\ref{sec:experiment}. Finally, we 
conclude our article in~\ref{sec:conclusion}. 
