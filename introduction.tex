\section{Introduction}
\label{sec:introduction}

%  Multiband background
The FCC has approved the use of broadband services in the white spaces of 
UHF TV bands, which were formerly exclusively licensed to television broadcasters.
These white space bands are now available for unlicensed public use, enabling the
deployment of wireless access networks. These white space bands operate in available 
channels from 54-806 MHz, having a far greater propagation range than WiFi bands for 
similar transmission power~\cite{balanis2012antenna}. 
% Improve existing cells
Thus, white space bands could greatly complement the existing WiFi wireless network with 
a large area. The users in the propagation range of the access point with white space radios 
has the options to associate with either the WiFi channel of its cell or the white space 
channel.



% Multi-user diersity
The users in multiple locations under the coverage of both the WiFi and white space have {\it user 
diversity}. The term user diversity represents the same frequency band at the same time can offer 
different transmission qualities to different users due to their difference in transceiver design, 
geographic location, etc.
% Explain the spectral diversity temporal diversity
The user diversity comes from two types of diversity gains. One is the temporal diversity which 
is caused by the environment variation. Another is the spectral diversity which represents the 
transmission conditions varies across channels. 
In some moderate number of users, the sum capacity of the fading channel is greater than 
the sum capacity of a nonfading channel. In the fading channels, the sum capacity of users 
increase with the number of users in the system~\cite{viswanath2002opportunistic,gan2014multiple}. 


% White space limited resource
%Moreover, the number of available white space channels are restricted by FCC. In sparse rural area,
%there are more free channels in white space. Inversely, few available channels are located in dense 
%populated area. The question {\it how limited white space resource befinet the access of the users 
%in these diversity scenarios?}

Previous work studied the multi user setting with a single channel~\cite{tan2010distributed}. Spectral diversity 
is isolated for a single user in~\cite{shu2009throughput}. In~\cite{liu2013stay}, multi-user dynamic channel access is 
proposed jointly consider the temporal and spectral diversity in a multichannel model. However, 
none of these works address the channel association problem in multiband scenario.

The larger propagation range of white space channels adapt channel association of users located 
in large area through time division. When the users distributed in a large area, the temporal 
diversity and spectral diversity become the key issues of white space applications.
Previous work~\cite{pcuiwinmee} studied the white space application in access network deployment 
with spectral diversity. However, these works fails to leverage the white space frequency in 
multi-user diversity in both spectral and temporal scenarios.

% Extreme cases and the middle of number of channels
In sparse rural areas, plenty of white space channels are able to deploy new white space network. However, 
in dense area, few white space channels are available for new network deployment, such as none white 
space channel is available in New York downtown~\cite{googlespectrum}. The carrier have to use WiFi 
channels to deploy wireless networks in the dense area without any available white space channel. 
Other than these two extreme cases, most areas of major cities in the United States have one to eight 
white space channels~\cite{googlespectrum}. Exploiting these limited white space resource to improve 
the WiFi network in dense area is a perspective option for the dense areas.

% Power saving, extreme cases
The white space frequencies offer more wireless capacity and convenience of access across large 
area. When the traffic demands of the users are relatively low, a single white space channels satisfy 
all the user. Then, the WiFi radios could be turned off for saving power. On the other side, 
when the users have high traffic demand, all the radios, WiFi and white space, have to be operated to serve the users. 
However, traffic demands generally come across somewhere between these extremes.
Thus, the question comes out,a{\it in what degree the white space help to reduce the power consumptions of 
an existing WiFi mesh?} 

% Traffic Demand
In this work, we study channel schedule in a multi-user multiband setting, where users are not 
fully backlogged, traffic demand follow a certain arrival process. We focus on the effect of channel 
schedule of each user between the WiFi or white space band. 

% On demand consideration

% Paper contributions
In particular, the main contributions of our work are as follows:
\begin{itemize}
\item We perform long-term in-field measurements in neighborhoods, campus, downtown business building, 
and urban business buildings. Through the measurements, we estimate the achieved channel capacity in 
these area.
\item We formulate the channel resource allocation problem as a queuing system. Based on previous works, we 
propose the response time estimation methods of multiple scenarios for the heterogeneous network.
\item We develop a Greedy Server-side Replace algorithm to allocate the channel capacity for the users with minimum 
power consumption.
\item We perform measurement-driven simulation to analyze various scenarios of channel resource and users distribution. 
Our results shows that the white space bands reduce the power consumption in typical city environment by FIXME.
\end{itemize}

The rest of the paper 
